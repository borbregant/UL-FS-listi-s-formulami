\documentclass[a4paper,10pt]{article}
\usepackage[utf8]{inputenc}
\usepackage{amsmath,amssymb,amsthm}
\usepackage{multicol}
\usepackage{geometry}
\usepackage[most]{tcolorbox}
\usepackage{graphicx}
\usepackage{array}

\geometry{margin=1cm}
\setlength{\parskip}{0pt}
\setlength{\parindent}{0pt}
\renewcommand{\arraystretch}{1.2}

% Define a custom box style
\newtcolorbox{formulaBox}[1]{
	colback=white,
	colframe=black,
	boxrule=0.8pt,
	arc=4pt,
	left=6pt,
	right=6pt,
	top=2pt,
	bottom=2pt,
	enhanced,
	title={\scshape #1}, % Title for the box
	coltitle=black,
	fonttitle=\sffamily\small,
	attach title to upper=\\[-2mm],
}

\thispagestyle{empty}
\pagestyle{empty}

\begin{document}
\vspace*{5mm}
\begin{multicols}{2}

\begin{formulaBox}{Odvodi in inverse funkcije}
\[
(\arcsin x)' = \frac{1}{\sqrt{1-x^2}}\]
\[(\arccos x)' = -\frac{1}{\sqrt{1-x^2}}\]
\[(\arctan x)' = \frac{1}{1+x^2}\]
\end{formulaBox}

\begin{formulaBox}{Kot med premicama}
\[
\cos \varphi = |\frac{k_1 - k_2}{1+k_1 k_2}|
\]
\end{formulaBox}

\begin{formulaBox}{Stacionarne točke}
	\\
$f'(x_0)=0$. Če $f''(x_0)>0$ → minimum, če $f''(x_0)<0$ → maksimum, če $f''(x_0)=0$ → test višjih odvodov ali infleksijska točka.
\end{formulaBox}

\begin{formulaBox}{Konveksnost / konkavnost}
	\\
$f''(x)>0$ → konveksna, $f''(x)<0$ → konkavna.
\end{formulaBox}

\begin{formulaBox}{Taylorjeva vrsta okrog $a$}
\[
f(x)=f(a)+f'(a)(x-a)+\tfrac{f''(a)}{2!}(x-a)^2+\cdots
\]
\end{formulaBox}

\begin{formulaBox}{Per partes}
\[
\int u\,dv = uv - \int v\,du
\]
\end{formulaBox}

\begin{formulaBox}{Znani integrali}
\[
\int e^{ax}\sin(bx)\,dx = \frac{e^{ax}}{a^2+b^2}\big(a\sin(bx) - b\cos(bx)\big) + C
\]
\[
\int e^{ax}\cos(bx)\,dx = \frac{e^{ax}}{a^2+b^2}\big(a\cos(bx) + b\sin(bx)\big) + C
\]
\[
\int \frac{dx}{x^2+a^2} = \frac{1}{a}\arctan\!\left(\frac{x}{a}\right)+C
\]
\end{formulaBox}

\begin{formulaBox}{Parcialni ulomki}
	\\
\begin{tabular}{p{0.4\linewidth}p{0.4\linewidth}}
 Ulomek &  Parcialni razcep \\
\hline
$\displaystyle\frac{px+q}{(x-a)(x-b)}, a \ne b$ & $\displaystyle\frac{A}{x-a} + \frac{B}{x-b}$ \\[4mm]
$\displaystyle\frac{px+q}{(x-a)^2}$ & $\displaystyle\frac{A}{x-a} + \frac{B}{(x-a)^2}$ \\[4mm]
$\displaystyle\frac{px^2+qx+r}{(x-a)(x^2+bx+c)}$ & $\displaystyle\frac{A}{x-a} + \frac{Bx+C}{x^2+bx+c}$ \\[4mm]
\multicolumn{2}{p{\dimexpr.8\linewidth+4\tabcolsep\relax}}{%
    kjer se $x^2+bx+c$ se ne da razstaviti naprej.} \\
\end{tabular}
\end{formulaBox}

\begin{formulaBox}{Trigonometrične formule}
\[
\sin^2 x = \frac{1-\cos 2x}{2},\quad \cos^2 x = \frac{1+\cos 2x}{2}
\]
\end{formulaBox}

\begin{formulaBox}{Univerzalna substitucija}
\[t=\tan \tfrac{x}{2},\hspace{0.5em}
\sin x = \frac{2t}{1+t^2},\hspace{0.5em}
\cos x = \frac{1-t^2}{1+t^2},\hspace{0.5em}
dx=\frac{2}{1+t^2}dt
\]
\end{formulaBox}

\begin{formulaBox}{Prostornina in površina vrtenine, dolžina loka}
\[
V = \pi\int_a^b f(x)^2 dx\]
\[
pl = 2\pi\int_a^b f(x)\sqrt{1+(f'(x))^2}\,dx
\]
\[L=\int_a^b \sqrt{1+(f'(x))^2}\,dx\]
\end{formulaBox}

\begin{formulaBox}{Normala in tangentna ravnina}
	\\
Normala $\vec{n}=(f_x,f_y,-1)$\\
Tangentna ravnina $z=f_x(x_0,y_0)x+f_y(x_0,y_0)y+d$.
\end{formulaBox}

\begin{formulaBox}{Ekstremi 2D}
	\\
$\det H_f>0, f_{xx}>0$: minimum\\
$\det H_f>0, f_{xx}<0$: maksimum\\
$\det H_f<0$: sedlo\\
$\det H_f=0$: ne vemo
\end{formulaBox}

\begin{formulaBox}{Inverzi matrik}
\[\begin{bmatrix}a&b\\c&d\end{bmatrix}^{-1}=\frac{1}{ad-bc}\begin{bmatrix}d&-b\\-c&a\end{bmatrix}
\]
\[
A^{-1} = \frac{1}{\det A}\,adj(A),\quad 
adj(A)=\big[(-1)^{i+j}D_{ij}\big]^T
\]
\end{formulaBox}

\begin{formulaBox}{Cramerjevo pravilo}
\[
x_i = \frac{D_i}{D},\quad D=\det(A),\quad D_i=\det(A \text{ z $i$-tim stolpcem $\vec b$})
\]
\end{formulaBox}

\begin{formulaBox}{Lastne vrednosti}
	\\
$\det(A-\lambda I)=0$, lastni vektorji rešijo $(A-\lambda I)\vec v=0$.
\end{formulaBox}

\begin{formulaBox}{Skalarni, vektorski in mešani produkt}
\[
\vec a\cdot \vec b = |a||b|\cos\varphi =a_1b_1+a_2b_2+a_3b_3,\quad |\vec a|=\sqrt{\vec a\cdot \vec a}
\]
\[
\vec a\times \vec b = 
\det\begin{bmatrix}
\vec i & \vec j & \vec k\\
a_1 & a_2 & a_3\\
b_1 & b_2 & b_3
\end{bmatrix},\quad
|\vec a\times \vec b|=|a||b|\sin\varphi
\]
\[
(\vec a\times \vec b)\cdot \vec c = \det[a,b,c]
\]
\end{formulaBox}

\begin{formulaBox}{Dvojni vektorski produkt, Lagrangeva identiteta}

\[
\vec (a\times\vec b)\times \vec c= (\vec a\cdot \vec c)\vec b-(\vec b\cdot \vec c)\vec a
\]

\[
|\vec a\times \vec b|^2=|\vec a|^2|\vec b|^2-(\vec a\cdot \vec b)^2
\]
\end{formulaBox}

\begin{formulaBox}{Razdalje}
    %\[d(A, B) = |\overrightarrow{AB}|\]
    \[d(T, p) = \frac{|\overrightarrow{T_0T} \times \vec{s}|}{|\vec{s}|}\]
    \[d(T, \Sigma) = \frac{|at_x + bt_y + ct_z - d|}{|\vec{n}|}\]
    \[d(p_1, p_2) = \frac{|(\overrightarrow{T_1T_2}, \vec{s_1}, \vec{s_2})|}{|\vec{s_1} \times \vec{s_2}|}\]
\end{formulaBox}

\end{multicols}
\begin{formulaBox}{Faktorizacija in defaktorizacija}
\begin{align*}
\sin \alpha + \sin \beta &= 2 \sin \frac{\alpha + \beta}{2} \cos \frac{\alpha - \beta}{2} & \sin \alpha \cos \beta &= \frac{1}{2}(\sin(\alpha + \beta) + \sin(\alpha - \beta)) \\
\cos \alpha + \cos \beta &= 2 \cos \frac{\alpha + \beta}{2} \cos \frac{\alpha - \beta}{2} & \cos \alpha \cos \beta &= \frac{1}{2}(\cos(\alpha + \beta) + \cos(\alpha - \beta)) \\
\cos \alpha - \cos \beta &= -2 \sin \frac{\alpha + \beta}{2} \sin \frac{\alpha - \beta}{2} & \sin \alpha \sin \beta &= \frac{1}{2}(\cos(\alpha - \beta) - \cos(\alpha + \beta))
\end{align*}
\end{formulaBox}
\end{document}